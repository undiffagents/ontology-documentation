\chapter{Overview}

We are presenting the ontology which drives the data gathering and integration done as part of the project, \emph{Towards Undifferentiated Cognitive Agents: Determining Gaps in Comprehension},\footnote{See \url{https://daselab.cs.ksu.edu/projects/afosr-cogagents}.} funded by the Air Force Office of Scientific Research under award number FA9550-18-1-0386. It is a collaborative effort with the projects \emph{Toward Undifferentiated Cognitive Agents for Diverse Specializations} (Air Force Research Laboratory, PIs Chris Myers (RH) and Benji Maruyama (RX)) and \emph{Toward Undifferentiated Cognitive Agents: Translating Instructions to Knowledge} (Drexel University, PI: Dario Salvucci).


Development of the ontology was a collaborative effort and was carried out using the principles laid out in, e.g., \cite{moe-chap}. The modeling team included domain experts, data experts, software developers, and ontology engineers. 

The ontology has, in particular, be developed as a \emph{modular} ontology \cite{moe-chap,HitzlerS18} based on ontology design patterns \cite{HGJKP2016}.\footnote{See \url{https://daselab.cs.ksu.edu/content/modular-ontology-engineering-portal} for pointers to further resources on the approach.} This means, in a nutshell, that we first identified key terms relating to the data content and expert perspectives on the domain to be modeled, and then developed ontology modules for these terms. The resulting modules, which were informed by corresponding ontology design patterns, are listed and discussed in Chapter \ref{sec:mods}. The Uagent Ontology, assembled from these modules, is then persented in Chapter \ref{chap:ontology}.

For background regarding Semantic Web standards, in particular the Web Ontology Language OWL, including its relation to description logics, we refer the reader to \cite{owl2-primer, FOST}.


